\documentclass[11pt]{article}
\usepackage[a4paper, margin=2.5cm]{geometry}
\usepackage{amsmath, amssymb, graphicx, hyperref}
\usepackage{authblk}
\usepackage{physics}
\usepackage{bm}
\usepackage{float}
\usepackage{cite}
\input{macros.tex}

\title{Taquiones Oscuros y Computación Cuántica No Local:\\ Un Marco para Qubits Estables y Redes Distribuidas}
\author{}
\date{21 de junio de 2025}

\begin{document}
\maketitle

\begin{abstract}
Este trabajo presenta un modelo teórico que conecta cosmología cuántica, simetría $\mathcal{PT}$ y geometría no conmutativa para obtener qubits estables y teleportación cuántica no local. Se proponen simulaciones numéricas y predicciones falsables para validar el enfoque.
\end{abstract}

\section{Introducción}
El entrelazamiento cuántico global es uno de los retos fundamentales de la computación cuántica escalable. En este trabajo, proponemos un modelo de \textit{taquiones oscuros} (TO) que actúa como campo de fondo para estabilizar qubits no hermíticos bajo simetría $\mathcal{PT}$ y permite teleportación cuántica sin canal físico.

\section{Modelo Teórico}
El lagrangiano TO propuesto es:
\[
\mathcal{L}_{TO} = -V(\phi) \sqrt{1 - (\partial_\mu \phi)^2} + g_x \phi^2 \chi^\dagger \chi + \delta \phi^6
\]
donde $V(\phi)$ incluye correcciones polinómicas cuánticas.

\section{Simetría $\mathcal{PT}$ y Qubits No Hermíticos}
El Hamiltoniano cuántico efectivo es:
\[
H = \begin{pmatrix} 0 & 1 \\ 1 & 0 \end{pmatrix} + i\gamma \begin{pmatrix} 1 & 0 \\ 0 & -1 \end{pmatrix}
\]
que se estudia mediante simulaciones con Qiskit.

\section{Geometría No Conmutativa}
Utilizamos un tensor $\theta^{\mu\nu}$ para simular redes no locales. La teleportación se modela con:
\[
|\Psi\rangle = \frac{1}{\sqrt{2}}(|00\rangle + |11\rangle) \otimes e^{-\frac{1}{2}p_\mu \theta^{\mu\nu} p_\nu}
\]

\section{Resultados Numéricos}
Se obtuvieron fidelidades $> 99.95\%$ y predicciones falsables como $r_B \approx 2.38 \times 10^{-3}$ en modos B del CMB.

\section{Conclusiones}
Este marco conecta cosmología, cuerdas y computación cuántica en un modelo testable. Próximos pasos incluyen validación experimental en hardware cuántico.

\bibliographystyle{unsrt}
\bibliography{references}

\end{document}
